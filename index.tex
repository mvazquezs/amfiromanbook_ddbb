% Options for packages loaded elsewhere
% Options for packages loaded elsewhere
\PassOptionsToPackage{unicode}{hyperref}
\PassOptionsToPackage{hyphens}{url}
\PassOptionsToPackage{dvipsnames,svgnames,x11names}{xcolor}
%
\documentclass[
  12pt,
  letterpaper,
  DIV=11,
  numbers=noendperiod]{scrreprt}
\usepackage{xcolor}
\usepackage{amsmath,amssymb}
\setcounter{secnumdepth}{5}
\usepackage{iftex}
\ifPDFTeX
  \usepackage[T1]{fontenc}
  \usepackage[utf8]{inputenc}
  \usepackage{textcomp} % provide euro and other symbols
\else % if luatex or xetex
  \usepackage{unicode-math} % this also loads fontspec
  \defaultfontfeatures{Scale=MatchLowercase}
  \defaultfontfeatures[\rmfamily]{Ligatures=TeX,Scale=1}
\fi
\usepackage{lmodern}
\ifPDFTeX\else
  % xetex/luatex font selection
  \setmainfont[]{LMRoman10-Regular}
\fi
% Use upquote if available, for straight quotes in verbatim environments
\IfFileExists{upquote.sty}{\usepackage{upquote}}{}
\IfFileExists{microtype.sty}{% use microtype if available
  \usepackage[]{microtype}
  \UseMicrotypeSet[protrusion]{basicmath} % disable protrusion for tt fonts
}{}
\makeatletter
\@ifundefined{KOMAClassName}{% if non-KOMA class
  \IfFileExists{parskip.sty}{%
    \usepackage{parskip}
  }{% else
    \setlength{\parindent}{0pt}
    \setlength{\parskip}{6pt plus 2pt minus 1pt}}
}{% if KOMA class
  \KOMAoptions{parskip=half}}
\makeatother
% Make \paragraph and \subparagraph free-standing
\makeatletter
\ifx\paragraph\undefined\else
  \let\oldparagraph\paragraph
  \renewcommand{\paragraph}{
    \@ifstar
      \xxxParagraphStar
      \xxxParagraphNoStar
  }
  \newcommand{\xxxParagraphStar}[1]{\oldparagraph*{#1}\mbox{}}
  \newcommand{\xxxParagraphNoStar}[1]{\oldparagraph{#1}\mbox{}}
\fi
\ifx\subparagraph\undefined\else
  \let\oldsubparagraph\subparagraph
  \renewcommand{\subparagraph}{
    \@ifstar
      \xxxSubParagraphStar
      \xxxSubParagraphNoStar
  }
  \newcommand{\xxxSubParagraphStar}[1]{\oldsubparagraph*{#1}\mbox{}}
  \newcommand{\xxxSubParagraphNoStar}[1]{\oldsubparagraph{#1}\mbox{}}
\fi
\makeatother

\usepackage{color}
\usepackage{fancyvrb}
\newcommand{\VerbBar}{|}
\newcommand{\VERB}{\Verb[commandchars=\\\{\}]}
\DefineVerbatimEnvironment{Highlighting}{Verbatim}{commandchars=\\\{\}}
% Add ',fontsize=\small' for more characters per line
\usepackage{framed}
\definecolor{shadecolor}{RGB}{241,243,245}
\newenvironment{Shaded}{\begin{snugshade}}{\end{snugshade}}
\newcommand{\AlertTok}[1]{\textcolor[rgb]{0.68,0.00,0.00}{#1}}
\newcommand{\AnnotationTok}[1]{\textcolor[rgb]{0.37,0.37,0.37}{#1}}
\newcommand{\AttributeTok}[1]{\textcolor[rgb]{0.40,0.45,0.13}{#1}}
\newcommand{\BaseNTok}[1]{\textcolor[rgb]{0.68,0.00,0.00}{#1}}
\newcommand{\BuiltInTok}[1]{\textcolor[rgb]{0.00,0.23,0.31}{#1}}
\newcommand{\CharTok}[1]{\textcolor[rgb]{0.13,0.47,0.30}{#1}}
\newcommand{\CommentTok}[1]{\textcolor[rgb]{0.37,0.37,0.37}{#1}}
\newcommand{\CommentVarTok}[1]{\textcolor[rgb]{0.37,0.37,0.37}{\textit{#1}}}
\newcommand{\ConstantTok}[1]{\textcolor[rgb]{0.56,0.35,0.01}{#1}}
\newcommand{\ControlFlowTok}[1]{\textcolor[rgb]{0.00,0.23,0.31}{\textbf{#1}}}
\newcommand{\DataTypeTok}[1]{\textcolor[rgb]{0.68,0.00,0.00}{#1}}
\newcommand{\DecValTok}[1]{\textcolor[rgb]{0.68,0.00,0.00}{#1}}
\newcommand{\DocumentationTok}[1]{\textcolor[rgb]{0.37,0.37,0.37}{\textit{#1}}}
\newcommand{\ErrorTok}[1]{\textcolor[rgb]{0.68,0.00,0.00}{#1}}
\newcommand{\ExtensionTok}[1]{\textcolor[rgb]{0.00,0.23,0.31}{#1}}
\newcommand{\FloatTok}[1]{\textcolor[rgb]{0.68,0.00,0.00}{#1}}
\newcommand{\FunctionTok}[1]{\textcolor[rgb]{0.28,0.35,0.67}{#1}}
\newcommand{\ImportTok}[1]{\textcolor[rgb]{0.00,0.46,0.62}{#1}}
\newcommand{\InformationTok}[1]{\textcolor[rgb]{0.37,0.37,0.37}{#1}}
\newcommand{\KeywordTok}[1]{\textcolor[rgb]{0.00,0.23,0.31}{\textbf{#1}}}
\newcommand{\NormalTok}[1]{\textcolor[rgb]{0.00,0.23,0.31}{#1}}
\newcommand{\OperatorTok}[1]{\textcolor[rgb]{0.37,0.37,0.37}{#1}}
\newcommand{\OtherTok}[1]{\textcolor[rgb]{0.00,0.23,0.31}{#1}}
\newcommand{\PreprocessorTok}[1]{\textcolor[rgb]{0.68,0.00,0.00}{#1}}
\newcommand{\RegionMarkerTok}[1]{\textcolor[rgb]{0.00,0.23,0.31}{#1}}
\newcommand{\SpecialCharTok}[1]{\textcolor[rgb]{0.37,0.37,0.37}{#1}}
\newcommand{\SpecialStringTok}[1]{\textcolor[rgb]{0.13,0.47,0.30}{#1}}
\newcommand{\StringTok}[1]{\textcolor[rgb]{0.13,0.47,0.30}{#1}}
\newcommand{\VariableTok}[1]{\textcolor[rgb]{0.07,0.07,0.07}{#1}}
\newcommand{\VerbatimStringTok}[1]{\textcolor[rgb]{0.13,0.47,0.30}{#1}}
\newcommand{\WarningTok}[1]{\textcolor[rgb]{0.37,0.37,0.37}{\textit{#1}}}

\usepackage{longtable,booktabs,array}
\usepackage{calc} % for calculating minipage widths
% Correct order of tables after \paragraph or \subparagraph
\usepackage{etoolbox}
\makeatletter
\patchcmd\longtable{\par}{\if@noskipsec\mbox{}\fi\par}{}{}
\makeatother
% Allow footnotes in longtable head/foot
\IfFileExists{footnotehyper.sty}{\usepackage{footnotehyper}}{\usepackage{footnote}}
\makesavenoteenv{longtable}
\usepackage{graphicx}
\makeatletter
\newsavebox\pandoc@box
\newcommand*\pandocbounded[1]{% scales image to fit in text height/width
  \sbox\pandoc@box{#1}%
  \Gscale@div\@tempa{\textheight}{\dimexpr\ht\pandoc@box+\dp\pandoc@box\relax}%
  \Gscale@div\@tempb{\linewidth}{\wd\pandoc@box}%
  \ifdim\@tempb\p@<\@tempa\p@\let\@tempa\@tempb\fi% select the smaller of both
  \ifdim\@tempa\p@<\p@\scalebox{\@tempa}{\usebox\pandoc@box}%
  \else\usebox{\pandoc@box}%
  \fi%
}
% Set default figure placement to htbp
\def\fps@figure{htbp}
\makeatother


% definitions for citeproc citations
\NewDocumentCommand\citeproctext{}{}
\NewDocumentCommand\citeproc{mm}{%
  \begingroup\def\citeproctext{#2}\cite{#1}\endgroup}
\makeatletter
 % allow citations to break across lines
 \let\@cite@ofmt\@firstofone
 % avoid brackets around text for \cite:
 \def\@biblabel#1{}
 \def\@cite#1#2{{#1\if@tempswa , #2\fi}}
\makeatother
\newlength{\cslhangindent}
\setlength{\cslhangindent}{1.5em}
\newlength{\csllabelwidth}
\setlength{\csllabelwidth}{3em}
\newenvironment{CSLReferences}[2] % #1 hanging-indent, #2 entry-spacing
 {\begin{list}{}{%
  \setlength{\itemindent}{0pt}
  \setlength{\leftmargin}{0pt}
  \setlength{\parsep}{0pt}
  % turn on hanging indent if param 1 is 1
  \ifodd #1
   \setlength{\leftmargin}{\cslhangindent}
   \setlength{\itemindent}{-1\cslhangindent}
  \fi
  % set entry spacing
  \setlength{\itemsep}{#2\baselineskip}}}
 {\end{list}}
\usepackage{calc}
\newcommand{\CSLBlock}[1]{\hfill\break\parbox[t]{\linewidth}{\strut\ignorespaces#1\strut}}
\newcommand{\CSLLeftMargin}[1]{\parbox[t]{\csllabelwidth}{\strut#1\strut}}
\newcommand{\CSLRightInline}[1]{\parbox[t]{\linewidth - \csllabelwidth}{\strut#1\strut}}
\newcommand{\CSLIndent}[1]{\hspace{\cslhangindent}#1}



\setlength{\emergencystretch}{3em} % prevent overfull lines

\providecommand{\tightlist}{%
  \setlength{\itemsep}{0pt}\setlength{\parskip}{0pt}}



 


\KOMAoption{captions}{tableheading}
\makeatletter
\@ifpackageloaded{bookmark}{}{\usepackage{bookmark}}
\makeatother
\makeatletter
\@ifpackageloaded{caption}{}{\usepackage{caption}}
\AtBeginDocument{%
\ifdefined\contentsname
  \renewcommand*\contentsname{Table of contents}
\else
  \newcommand\contentsname{Table of contents}
\fi
\ifdefined\listfigurename
  \renewcommand*\listfigurename{List of Figures}
\else
  \newcommand\listfigurename{List of Figures}
\fi
\ifdefined\listtablename
  \renewcommand*\listtablename{List of Tables}
\else
  \newcommand\listtablename{List of Tables}
\fi
\ifdefined\figurename
  \renewcommand*\figurename{Figure}
\else
  \newcommand\figurename{Figure}
\fi
\ifdefined\tablename
  \renewcommand*\tablename{Table}
\else
  \newcommand\tablename{Table}
\fi
}
\@ifpackageloaded{float}{}{\usepackage{float}}
\floatstyle{ruled}
\@ifundefined{c@chapter}{\newfloat{codelisting}{h}{lop}}{\newfloat{codelisting}{h}{lop}[chapter]}
\floatname{codelisting}{Listing}
\newcommand*\listoflistings{\listof{codelisting}{List of Listings}}
\makeatother
\makeatletter
\makeatother
\makeatletter
\@ifpackageloaded{caption}{}{\usepackage{caption}}
\@ifpackageloaded{subcaption}{}{\usepackage{subcaption}}
\makeatother
\usepackage{bookmark}
\IfFileExists{xurl.sty}{\usepackage{xurl}}{} % add URL line breaks if available
\urlstyle{same}
\hypersetup{
  pdftitle={Anàlisi Quantitativa d'Amfiteatres Romans},
  pdfauthor={Miquel Vàzquez-Santiago},
  colorlinks=true,
  linkcolor={blue},
  filecolor={Maroon},
  citecolor={Blue},
  urlcolor={Blue},
  pdfcreator={LaTeX via pandoc}}


\title{Anàlisi Quantitativa d'Amfiteatres Romans}
\author{Miquel Vàzquez-Santiago}
\date{September 30, 2025}
\begin{document}
\maketitle

\renewcommand*\contentsname{Table of contents}
{
\hypersetup{linkcolor=}
\setcounter{tocdepth}{2}
\tableofcontents
}

\bookmarksetup{startatroot}

\chapter*{Prefaci}\label{prefaci}
\addcontentsline{toc}{chapter}{Prefaci}

\markboth{Prefaci}{Prefaci}

Benvingut a aquest estudi sobre els amfiteatres de l'antiga Roma, una
exploració que neix de la fascinació per aquestes grans obres
d'enginyeria i de la voluntat d'entendre-les a través d'una nova
perspectiva: l'anàlisi quantitativa.

Durant segles, aquests edificis han estat estudiats des de
l'arqueologia, la història i la història de l'art. Hem après sobre els
espectacles que acollien, les tècniques constructives que els van fer
possibles i el seu paper com a centres de la vida social i política de
les ciutats romanes. No obstant això, moltes preguntes sobre les seves
dimensions i proporcions a gran escala romanen obertes.

Responien tots els amfiteatres a un únic cànon arquitectònic? Fins a
quin punt variaven les seves dimensions segons la província on es
construïen? Podem identificar ``famílies'' o tipologies d'amfiteatres a
partir de la seva mètrica?

Aquest treball intenta abordar aquestes qüestions mitjançant la
recopilació i l'anàlisi estadística d'una base de dades dimensional.
Lluny de pretendre substituir els estudis tradicionals, l'objectiu és
complementar-los, oferint una visió panoràmica que permeti identificar
tendències i patrons a una escala que l'estudi de cas aïllat no pot
abastar.

Espero que el lector trobi en aquestes pàgines una aproximació
interessant i enriquidora a un dels llegats més impressionants del món
romà.

\emph{Miquel Vàzquez-Santiago}

\emph{Setembre de 2025}

\begin{figure}[H]

{\centering \includegraphics[width=0.8\linewidth,height=\textheight,keepaspectratio]{cover.png}

}

\caption{Imatge de coberta}

\end{figure}%

\bookmarksetup{startatroot}

\chapter{Abstract}\label{abstract}

Els amfiteatres romans són una de les manifestacions més representatives
de l'arquitectura i l'enginyeria de l'antiga Roma. Aquest treball
presenta una anàlisi quantitativa de les dimensions d'aquests edificis a
tot l'Imperi, amb l'objectiu de caracteritzar-ne la mètrica, identificar
possibles diferències regionals i explorar l'existència de tipologies
basades en les seves proporcions arquitectòniques.

Per a això, s'ha compilat una base de dades a partir de fonts
publicades, principalment el catàleg de Jean-Claude Golvin (Golvin
(1988)). Les dades absents, un problema inherent a les fonts
arqueològiques, s'han tractat mitjançant mètodes d'imputació no
paramètrics basats en \emph{Random Forest} (\texttt{missForest}).
L'anàlisi estadística posterior ha inclòs tant l'estadística descriptiva
per caracteritzar les variables dimensionals, com proves inferencials no
paramètriques (test de Kruskal-Wallis i test de Dunn) per comparar les
dimensions entre les diferents províncies romanes.

Els resultats revelen una considerable variabilitat dimensional, tot i
que s'han pogut identificar diferències estadísticament significatives
entre províncies. Aquestes diferències regionals suggereixen que, lluny
d'un model arquitectònic homogeni, existien adaptacions locals o
``escoles'' constructives. L'anàlisi de les proporcions internes dels
edificis no ha permès, però, definir tipologies discretes, sinó més
aviat un continu de solucions formals.

En conclusió, aquest estudi demostra la utilitat de l'enfocament
quantitatiu per a l'estudi de l'arquitectura romana. Proporciona una
base empírica que matisa la noció d'un cànon arquitectònic rígid i
reforça la importància dels contextos regionals en la materialització
dels projectes constructius a l'Imperi Romà.

\begin{center}\rule{0.5\linewidth}{0.5pt}\end{center}

\textbf{Paraules clau:} amfiteatre romà, arquitectura, anàlisi
quantitativa, arqueologia computacional, províncies romanes, imputació
de dades.

\bookmarksetup{startatroot}

\chapter{Introducció}\label{introducciuxf3}

Els amfiteatres romans, escenaris de jocs gladiadors, caceres d'animals
(\texttt{venationes}) i espectacles públics, constitueixen una de les
expressions arquitectòniques més emblemàtiques i perdurables de l'Imperi
Romà. La seva distribució geogràfica, des de les províncies occidentals
fins a l'Orient, i la seva variabilitat dimensional reflecteixen la
complexitat de l'enginyeria romana i la seva adaptació a contextos
locals.

Tot i que nombrosos amfiteatres han estat objecte d'estudis monogràfics,
encara existeix una manca d'anàlisis quantitatives a gran escala que
comparin sistemàticament les seves característiques dimensionals. Aquest
treball busca omplir aquest buit mitjançant la creació i l'anàlisi d'una
base de dades que aglutina informació de fonts diverses, principalment
els treballs de Golvin (1988) i altres de més recents.

\section{Objectius i Preguntes de
Recerca}\label{objectius-i-preguntes-de-recerca}

Aquest estudi té com a objectiu principal analitzar i comparar les
dimensions dels amfiteatres romans per entendre millor els patrons
constructius i les possibles relacions entre la seva mida i la seva
ubicació geogràfica.

Les preguntes de recerca que guien aquest treball són:

\begin{enumerate}
\def\labelenumi{\arabic{enumi}.}
\tightlist
\item
  \textbf{Caracterització dimensional:} Quines són les dimensions
  típiques (i la seva variabilitat) dels amfiteatres romans pel que fa a
  l'arena, la càvea i les mides generals?
\item
  \textbf{Diferències provincials:} Existeixen diferències
  estadísticament significatives en les dimensions dels amfiteatres
  segons la província romana on es van construir?
\item
  \textbf{Identificació de tipologies:} Podem identificar grups o
  tipologies d'amfiteatres basant-nos en les proporcions entre les seves
  dimensions (p.~ex., la ràtio entre l'arena i l'edifici)?
\end{enumerate}

\section{Estructura del Treball}\label{estructura-del-treball}

Per respondre a aquestes preguntes, aquest treball s'estructura en els
següents capítols. El capítol de \textbf{Material i Mètodes} descriu les
fonts de dades i la metodologia estadística emprada. El capítol de
\textbf{Resultats} presenta les anàlisis descriptives i inferencials
dutes a terme. Finalment, el capítol de \textbf{Conclusions} discuteix
la interpretació dels resultats en el seu context històric i
arquitectònic, i apunta a futures línies d'investigació.

\bookmarksetup{startatroot}

\chapter{Material i Mètodes}\label{material-i-muxe8todes}

En aquest capítol es detallen els procediments utilitzats per a la
recopilació de dades, la seva preparació i l'anàlisi estadística duta a
terme per respondre a les preguntes de recerca plantejades.

\section{Fonts de Dades i Construcció de la Base de
Dades}\label{fonts-de-dades-i-construcciuxf3-de-la-base-de-dades}

La base de dades d'aquest estudi (\texttt{amfiroman\_ddbb}) és el
resultat d'un procés de compilació de fonts publicades. La font
principal és el catàleg d'amfiteatres de Jean-Claude Golvin (Golvin
(1988)), una referència fonamental en aquest camp. Aquesta informació ha
estat complementada i contrastada amb altres treballs acadèmics i bases
de dades més recents per tal d'obtenir un conjunt de dades el més
complet i actualitzat possible.

Les variables recollides per a cada amfiteatre inclouen:

\begin{itemize}
\tightlist
\item
  \textbf{Identificadors:} Nom de la ciutat romana (\texttt{nom}), país
  actual (\texttt{pais}).
\item
  \textbf{Geogràfics:} Província romana (\texttt{provincia\_romana}).
\item
  \textbf{Dimensionals (en metres):} \texttt{amplada\_general},
  \texttt{alcada\_general}, \texttt{amplada\_arena},
  \texttt{alcada\_arena}, i \texttt{amplada\_cavea}.
\end{itemize}

\section{Processament i Preparació de les
Dades}\label{processament-i-preparaciuxf3-de-les-dades}

Tot el processament i anàlisi de dades s'ha realitzat amb el llenguatge
de programació R (R Core Team 2023) i l'entorn RStudio. Els informes
s'han generat amb Quarto (Allaire et al. 2022). Per a la manipulació de
dades i la visualització es van utilitzar principalment els paquets
\texttt{dplyr} (Wickham et al. 2023) i \texttt{ggplot2} (Wickham 2016).
Les proves estadístiques es van realitzar amb funcions del paquet
\texttt{rstatix} (Kassambara 2023) i la imputació de dades amb
\texttt{mice} (van Buuren and Groothuis-Oudshoorn 2011) i
\texttt{missForest} (Stekhoven and Bühlmann 2012).

\subsection{Gestió de Dades Absents}\label{gestiuxf3-de-dades-absents}

Un dels reptes en treballar amb dades arqueològiques és la presència de
valors absents (\texttt{NA}), degut a l'estat de conservació desigual
dels monuments. Per tal de poder aplicar anàlisis multivariants sense
perdre un gran nombre de casos, es va optar per un mètode d'imputació de
dades.

El mètode seleccionat va ser \texttt{missForest} (Stekhoven and Bühlmann
2012), implementat a la funció \texttt{imputacio\_val\_perduts}. Aquest
és un algorisme no paramètric basat en \emph{Random Forest} que funciona
bé amb interaccions complexes i relacions no lineals entre variables,
sense fer suposicions sobre la distribució de les dades. Com a mètode de
reserva en cas d'error, es va establir la imputació per la
\texttt{mediana}.

L'impacte de la imputació en la distribució de les variables s'avaluarà
mitjançant la comparació de gràfics de densitat de les dades originals i
les imputades.

\section{Anàlisi Estadística}\label{anuxe0lisi-estaduxedstica}

L'estratègia d'anàlisi es va dissenyar per donar resposta a les
preguntes de recerca definides a la introducció, seguint el Pla
d'Anàlisi Estadística.

\subsection{Anàlisi Descriptiva (RQ1)}\label{anuxe0lisi-descriptiva-rq1}

Per respondre a la primera pregunta de recerca sobre la
\textbf{caracterització dimensional} dels amfiteatres, es va realitzar
una anàlisi descriptiva exhaustiva. Es van calcular els principals
estadístics (mitjana, mediana, desviació estàndard, quartils, etc.) per
a totes les variables dimensionals, utilitzant la funció
\texttt{tab\_summary}. A més, es van generar histogrames i gràfics de
densitat per visualitzar les seves distribucions.

\subsection{Anàlisi de Tipologies
(RQ3)}\label{anuxe0lisi-de-tipologies-rq3}

Per explorar la \textbf{identificació de tipologies}, es van crear noves
variables basades en les ràtios entre dimensions principals, com ara la
proporció entre l'amplada de l'arena i l'amplada general de l'edifici.
Aquestes noves variables de ràtio van ser analitzades descriptivament
per identificar possibles agrupacions.

\subsection{Anàlisi Inferencial (RQ2)}\label{anuxe0lisi-inferencial-rq2}

Per abordar la segona pregunta de recerca sobre les \textbf{diferències
provincials}, es va dur a terme una anàlisi inferencial. La selecció de
les proves estadístiques es va basar en el compliment del supòsit de
normalitat, avaluat amb el test de Shapiro-Wilk.

\begin{itemize}
\tightlist
\item
  \textbf{Comparació de grups:} Per a la comparació de les dimensions
  entre províncies romanes, es van utilitzar proves no paramètriques,
  atesa la distribució generalment no normal de les dades. Concretament:

  \begin{itemize}
  \tightlist
  \item
    \textbf{Test de Kruskal-Wallis:} Per comparar les mitjanes de més de
    dos grups (p.~ex., l'amplada general entre totes les províncies).
    Funció: \texttt{amphi\_stat\_multitest}.
  \item
    \textbf{Test de Dunn:} Com a prova \emph{post-hoc} per identificar
    quines províncies específiques difereixen entre si, en cas que el
    test de Kruskal-Wallis fos significatiu.
  \item
    \textbf{Ajust de p-valor:} Els p-valors de les comparacions
    múltiples es van ajustar amb el mètode de Bonferroni per controlar
    la taxa d'error de tipus I. Funció: \texttt{amphi\_adjust\_pvalue}.
  \end{itemize}
\item
  \textbf{Test de Wilcoxon:} Per a comparacions específiques entre dos
  grups. Funció: \texttt{amphi\_stat\_test}.
\end{itemize}

\bookmarksetup{startatroot}

\chapter{Resultats}\label{resultats}

En aquest capítol, presentem els resultats de l'anàlisi de les
dimensions dels amfiteatres romans. L'exposició s'estructura en tres
parts principals: una anàlisi descriptiva de les dades, el procés
d'imputació de valors perduts i, finalment, l'anàlisi inferencial per a
la comparació entre grups.

\section{Anàlisi Descriptiva}\label{anuxe0lisi-descriptiva}

Comencem amb una exploració de les dades per entendre les seves
característiques principals.

\subsection{Estadístiques
Descriptives}\label{estaduxedstiques-descriptives}

A la taula següent, es mostren les principals estadístiques descriptives
per a les variables dimensionals dels amfiteatres.

\begin{table}

\caption{\label{tbl-descriptius}Estadístiques descriptives de les
dimensions dels amfiteatres.}

\centering{

\begin{Shaded}
\begin{Highlighting}[]
\CommentTok{\# Aquí cridaries la teva funció tab\_summary.}
\CommentTok{\# Assegura\textquotesingle{}t que l\textquotesingle{}objecte \textquotesingle{}df\_amfiteatres\textquotesingle{} contingui les teves dades.}
\CommentTok{\# Per exemple:}
\CommentTok{\# tab\_summary(}
\CommentTok{\#   df = df\_amfiteatres,}
\CommentTok{\#   seleccio\_variables = starts\_with(\textquotesingle{}amplada\textquotesingle{}) | starts\_with(\textquotesingle{}alcada\textquotesingle{}),}
\CommentTok{\#   stats\_adicionals = TRUE,}
\CommentTok{\#   digits = 2}
\CommentTok{\# )}
\end{Highlighting}
\end{Shaded}

}

\end{table}%

Aquesta taula inclou el nombre d'observacions, la mitjana, la desviació
estàndard, els quartils i altres mètriques que ens informen sobre la
distribució i la dispersió de cada variable.

\subsection{Visualització de les
Dades}\label{visualitzaciuxf3-de-les-dades}

Per complementar la taula descriptiva, els següents gràfics mostren la
distribució de les principals variables dimensionals.

\begin{Shaded}
\begin{Highlighting}[]
\CommentTok{\# Aquí generaries els gràfics (p. ex., histogrames o boxplots amb ggplot2)}
\CommentTok{\# library(ggplot2)}
\CommentTok{\# }
\CommentTok{\# ggplot(df\_amfiteatres, aes(x = amplada\_general)) +}
\CommentTok{\#   geom\_histogram(bins = 15) +}
\CommentTok{\#   labs(title = "Distribució de l\textquotesingle{}Amplada General")}
\CommentTok{\# }
\CommentTok{\# ggplot(df\_amfiteatres, aes(x = amplada\_arena)) +}
\CommentTok{\#   geom\_histogram(bins = 15) +}
\CommentTok{\#   labs(title = "Distribució de l\textquotesingle{}Amplada de l\textquotesingle{}Arena")}
\CommentTok{\#}
\CommentTok{\# ggplot(df\_amfiteatres, aes(y = amplada\_general, x = provincia\_romana)) +}
\CommentTok{\#   geom\_boxplot() +}
\CommentTok{\#   labs(title = "Amplada General per Província") +}
\CommentTok{\#   theme(axis.text.x = element\_text(angle = 45, hjust = 1))}
\end{Highlighting}
\end{Shaded}

